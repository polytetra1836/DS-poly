\documentclass[UTF8]{ctexart}
\usepackage{geometry}
\geometry{margin=1.5cm, vmargin={0pt,1cm}}
\setlength{\topmargin}{-1cm}
\setlength{\paperheight}{29.7cm}
\setlength{\textheight}{25.3cm}

% useful packages.
\usepackage{amsfonts}
\usepackage{amsmath}
\usepackage{amssymb}
\usepackage{amsthm}
\usepackage{enumerate}
\usepackage{graphicx}
\usepackage{multicol}
\usepackage{fancyhdr}
\usepackage{layout}
\usepackage{listings}
\usepackage{float, caption}

\lstset{
    basicstyle=\ttfamily, basewidth=0.5em
}

% some common command
\newcommand{\dif}{\mathrm{d}}
\newcommand{\avg}[1]{\left\langle #1 \right\rangle}
\newcommand{\difFrac}[2]{\frac{\dif #1}{\dif #2}}
\newcommand{\pdfFrac}[2]{\frac{\partial #1}{\partial #2}}
\newcommand{\OFL}{\mathrm{OFL}}
\newcommand{\UFL}{\mathrm{UFL}}
\newcommand{\fl}{\mathrm{fl}}
\newcommand{\op}{\odot}
\newcommand{\Eabs}{E_{\mathrm{abs}}}
\newcommand{\Erel}{E_{\mathrm{rel}}}

\begin{document}

\pagestyle{fancy}
\fancyhead{}
\lhead{陈翔, 3230103619}
\chead{数据结构与算法项目作业}
\rhead{Nov.7th, 2024}

\section{四则运算表达式求值程序的设计思路}
按照要求,我在\textbf{\texttt{expression\_evaluator.h}}文件中编写4个函数:
\texttt{getPrecedenc}
\texttt{applyop}
\texttt{infixToPostfix}
\texttt{evaluatePostfix}
他们的功能依次是:规定运算符优先级、实现计算、转化中缀表达式为后缀表达式、计算后缀表达式。\par
按照上课的提示,我按照任课老师的思路对2个主要函数的功能进行实现,对表达式转化和计算时
用堆结构处理了运算符优先级和输入优先级,包括括号等不同情形。\par
转化时,将数字紧密排列,运算符和空格等在输入前加入空格,方便后续计算处理。\par
计算时,将表达式字符串按流读入,借助转化时写的空格合理断句。\par
为了支持科学计数法,我使用了std::stod()对整个转换后的表达式中数字进行了类型转化,这包括已经实现的
字符转化功能和科学计数的转化\textbf{\texttt{double}}类型功能。\par
针对负号,我将根据负号后的元素类型决定处理方式,当负号后接数字,将直接和数字合并为字符串,并用stod处理;
当负号后为括号,我会将表达式\textbf{op-()}强行转化为\textbf{op(0-())},该实现功能通过对负号后续的左括号进行标记,
在依次向栈中写入( - (后,我会向栈写入\#用于标记左括号,当读取到匹配的右括号,进行栈清除操作时程序会读到标记,程序将把标记和左括号正常删除,并令指标
i-1,等效于向表达式中加入了右括号,实现了表达式转化功能。\par
\section{输入规则}
参考现通用的计算器,我的计算器功能实现有以下约定:\par
1.平凡的括号是不合法的,比如1+()1,但是(1)是合法的。\par
2.2个以上连续的负号是不合法的,如果第一个被识别为减号,则第二个识别为负号,第三个继续识别为减号而报错;
如果第一个被识别为负号,则后续都会识别为减号并报错,因为我们往往不会发出使用多个连续负号的指令,除非
负号之间有括号相隔。\par
3.不存在正号的定义,所有的+都会识别为加号\par
4 科学计数法支持正负输入,且如果e后不接任何数,包括负数,将被默认为e的0次方,这是由stod函数的性质决定的。我们的程序功能是实现四则运算,
如果e后有完整的表达式,比如带上了括号,相当于在处理指数函数运算,因此我们的一维程序不处理这种复杂情况而认为这是不合法的。
因此科学计数法只支持直接接正负的数。\par
\section{测试函数的设计思路}
根据项目要求,我对以下情况或要求进行测试,给出了相应实例:\par
1.支持多重括号和四则运算;\par
2.支持有限位小数运算;\par
3.括号不匹配;\par
4.运算符连续使用;\par
5.表达式以运算符开头或结尾;\par
6.出现除数是 0 ;\par
7.识别负数;\par
8.支持科学计数法,包括正负幂、与其他运算符不会冲突 \par
9.识别负表达式,即负号后接括号\par
如果还有遗漏考虑的情况,可以持续输入表达式验证,直到输入exit退出。\par
\section{测试的结果}
以下是程序的输出内容:\par
固定测试样例:(1+2)*(1+2*(2+2)/(2-3)) 输出结果为 :-21\par
1.2*1.2 输出结果为 :1.44\par
(1) 输出结果为 :1\par
(2+2 输出结果为 :ILLEGAL\par
+2 输出结果为 :ILLEGAL\par
-2 输出结果为 :-2\par
2-2- 输出结果为 :ILLEGAL\par
1/0 输出结果为 :ILLEGAL\par
1++1.2 输出结果为 :ILLEGAL\par
1+-1.2 输出结果为 :-0.2\par
2e2-2 输出结果为 :198\par
2e-1+1 输出结果为 :1.2\par
2e-2*2 输出结果为 :0.04\par
1-(-1+1) 输出结果为 :1\par
1-(-1-(-1)) 输出结果为 :1\par
-(1+4) 输出结果为 :-5\par
请自主测试;输入表达式 (输入 'exit' 退出): 2e\par
2\par
请自主测试;输入表达式 (输入 'exit' 退出): 2e+1\par
3\par
请自主测试;输入表达式 (输入 'exit' 退出): 2e(1)\par
ILLEGAL\par
请自主测试;输入表达式 (输入 'exit' 退出): exit\par
退出程序.
\section{Bug 报告}
在主函数和测试过程中暂未发现明显问题,或者说,我们未实现的功能已经在规则中加以约定与解释。
\end{document}