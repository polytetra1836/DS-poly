\documentclass[UTF8]{ctexart}
\usepackage{geometry,CJKutf8}
\geometry{margin=1.5cm, vmargin={0pt,1cm}}
\setlength{\topmargin}{-1cm}
\setlength{\paperheight}{29.7cm}
\setlength{\textheight}{25.3cm}

% useful packages.
\usepackage{amsfonts}
\usepackage{amsmath}
\usepackage{amssymb}
\usepackage{amsthm}
\usepackage{enumerate}
\usepackage{graphicx}
\usepackage{multicol}
\usepackage{fancyhdr}
\usepackage{layout}
\usepackage{listings}
\usepackage{float, caption}

\lstset{
    basicstyle=\ttfamily, basewidth=0.5em
}

% some common command
\newcommand{\dif}{\mathrm{d}}
\newcommand{\avg}[1]{\left\langle #1 \right\rangle}
\newcommand{\difFrac}[2]{\frac{\dif #1}{\dif #2}}
\newcommand{\pdfFrac}[2]{\frac{\partial #1}{\partial #2}}
\newcommand{\OFL}{\mathrm{OFL}}
\newcommand{\UFL}{\mathrm{UFL}}
\newcommand{\fl}{\mathrm{fl}}
\newcommand{\op}{\odot}
\newcommand{\Eabs}{E_{\mathrm{abs}}}
\newcommand{\Erel}{E_{\mathrm{rel}}}

\begin{document}

\pagestyle{fancy}
\fancyhead{}
\lhead{陈翔, 3230103619}
\chead{数据结构与算法第四次作业}
\rhead{Oct.17th, 2024}

\section{测试程序的设计思路}

我首先创建了一个链表,向其中插入了元素,依次为从0到9的整数,测试了empty,push\_back,size等函数以及访问头尾元素的功能。\par
随后我用迭代器输出了一次数据,测试了迭代器功能,并向链表头、尾分别删除元素,再次输出链表数据。\par
完成上述操作,我清空了链表并检查元素个数,然后将链表恢复至第一次插入元素的状态。\par
最后,我分别检验了前置与后置自增、自减运算符的功能,以及赋值函数与赋值运算符,移动赋值函数与移动赋值运算符的功能。\par
每次操作,我都输出一次链表元素,对于创建新对象的操作,我检验对象地址和原地址以确定赋值不是浅赋值。\par
\section{测试的结果}
以下是程序输出内容:\\

List is empty: true\\
Pushing back elements 0 to 9:\\
0 1 2 3 4 5 6 7 8 9 \\
List is empty: false\\
List size after adding 10 elements: 10\\
First element: 0\\
Last element: 9\\
List elements: 0 1 2 3 4 5 6 7 8 9 \\
List size after popping front: 9\\
List size after popping back: 8\\
First element after popping front:1\\
Last element after popping back: 8\\
List size after inserting 1 and 5 at the front: 10\\
List elements after more insertions: 1 5 1 2 3 4 5 6 7 8 \\
List size after clearing: 0\\
We refresh the list to its original state\\
Using iterator and ++it:\\
0 1 2 3 4 5 6 7 8 \\
Using iterator and it++:\\
0 1 2 3 4 5 6 7 8 \\
Using iterator and --it(reverse):\\
8 7 6 5 4 3 2 1 0\\
Using iterator and it--(reverse) :\\
8 7 6 5 4 3 2 1 0\\
Copying list to a new list:\\
0 1 2 3 4 5 6 7 8 \\
The address of new list is different.\\
Assigning list to a new list:\\
0 1 2 3 4 5 6 7 8 \\
The address of new list is different.\\
Moving list to a new list:\\
0 1 2 3 4 5 6 7 8 \\
The address of new list is different.\\
Move-assigning list to a new list:\\
0 1 2 3 4 5 6 7 8 \\
The address of new list is different.\\
测试结果一切正常。

我用 valgrind 进行测试,发现没有发生内存泄露。

\section{bug报告}

我发现了一个List.h文件中的函数 bug:当erase函数接收的对象为空链表,编译完成后,可执行文件运行会在该栈生成后运行中断。\par
据我分析,它出现的原因是:
该函数没有对空指针进行检查,导致对空指针进行解引用而报错。在实际过程中,这一操作并不意外,我们有可能清空一个已经被清空的链表,
因此调用erase的函数都可能会出现同类bug。\par
对此,只需要在函数内部进行修改,对传入空指针的情况进行单独判断和返回。\par
我连续使用了2次clear函数,经测试,修改后的函数可以正常接收空链表,以此顺便解决clear等所有调用该函数的同类问题。\par
\end{document}

%%% Local Variables: 
%%% mode: latex
%%% TeX-master: t
%%% End: 
