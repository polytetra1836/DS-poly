\documentclass[UTF8]{ctexart}
\usepackage{geometry}
\geometry{margin=1.5cm, vmargin={0pt,1cm}}
\setlength{\topmargin}{-1cm}
\setlength{\paperheight}{29.7cm}
\setlength{\textheight}{25.3cm}

% useful packages.
\usepackage{amsfonts}
\usepackage{amsmath}
\usepackage{amssymb}
\usepackage{amsthm}
\usepackage{enumerate}
\usepackage{graphicx}
\usepackage{multicol}
\usepackage{fancyhdr}
\usepackage{layout}
\usepackage{listings}
\usepackage{float, caption}

\lstset{
    basicstyle=\ttfamily, basewidth=0.5em
}

% some common command
\newcommand{\dif}{\mathrm{d}}
\newcommand{\avg}[1]{\left\langle #1 \right\rangle}
\newcommand{\difFrac}[2]{\frac{\dif #1}{\dif #2}}
\newcommand{\pdfFrac}[2]{\frac{\partial #1}{\partial #2}}
\newcommand{\OFL}{\mathrm{OFL}}
\newcommand{\UFL}{\mathrm{UFL}}
\newcommand{\fl}{\mathrm{fl}}
\newcommand{\op}{\odot}
\newcommand{\Eabs}{E_{\mathrm{abs}}}
\newcommand{\Erel}{E_{\mathrm{rel}}}

\begin{document}

\pagestyle{fancy}
\fancyhead{}
\lhead{陈翔, 3230103619}
\chead{数据结构与算法第六题作业}
\rhead{Nov.9th, 2024}

\section{AVLT的转化和remove的实现}
按照要求,我将\textbf{\texttt{BST}}改为AVLT,在结构体中加入了子树高度,将\textbf{\texttt{private}}
的成员函数都改为\textbf{\texttt{protected}}。\par
我写入了\textbf{\texttt{BST}}balance功能,分别对于4种由子树高度变化1而导致子树不平衡的情况,分别用1次或
2次,左旋转和右旋转实现重新平衡的功能,每种旋转以及复合操作各写为了1个函数。\par
针对balance函数,我写了重载的\textbf{\texttt{isbalanced}}函数,对外开放的是接口,而
\textbf{\texttt{protected}}的用于检测树是否平衡,自上向下进行检测。\par
在\textbf{\texttt{remove}}和\textbf{\texttt{insert}}函数中,我在原有的函数末尾
调用\textbf{\texttt{balance}}函数,保证子树高度和平衡性能够随时更新,这种操作会引发栈的深度调用,不过在
严格的平衡要求下是安全的。\par
\section{测试的结果}
在运行\textbf{\texttt{test}}程序时,程序发生了类错误,在降低测试量后就不会发生这种情况。
经过检查,我消除了栈空间限制,使用了编译优化,随后程序正常运行,没有发生复制,运行时间未超时。\par
以下是程序最后输出结果:\par
Empty tree\par
------------------------------\par
Empty tree\par
2.26user 0.04system 0:02.30elapsed 99\%CPU (0avgtext+0avgdata 27008maxresident)k \par
0inputs+0outputs (0major+5994minor)pagefaults 0swaps\par
其中CPU占用率和运行时间与我的虚拟机操作环境相关,在其他设备运行时,运行时间在1s左右,CPU正常。
\end{document}
