\documentclass[UTF8]{ctexart}
\usepackage{geometry,CJKutf8}
\geometry{margin=1.5cm, vmargin={0pt,1cm}}
\setlength{\topmargin}{-1cm}
\setlength{\paperheight}{29.7cm}
\setlength{\textheight}{25.3cm}

% useful packages.
\usepackage{amsfonts}
\usepackage{amsmath}
\usepackage{amssymb}
\usepackage{amsthm}
\usepackage{enumerate}
\usepackage{graphicx}
\usepackage{multicol}
\usepackage{fancyhdr}
\usepackage{layout}
\usepackage{listings}
\usepackage{float, caption}

\lstset{
    basicstyle=\ttfamily, basewidth=0.5em
}

% some common command
\newcommand{\dif}{\mathrm{d}}
\newcommand{\avg}[1]{\left\langle #1 \right\rangle}
\newcommand{\difFrac}[2]{\frac{\dif #1}{\dif #2}}
\newcommand{\pdfFrac}[2]{\frac{\partial #1}{\partial #2}}
\newcommand{\OFL}{\mathrm{OFL}}
\newcommand{\UFL}{\mathrm{UFL}}
\newcommand{\fl}{\mathrm{fl}}
\newcommand{\op}{\odot}
\newcommand{\Eabs}{E_{\mathrm{abs}}}
\newcommand{\Erel}{E_{\mathrm{rel}}}

\begin{document}

\pagestyle{fancy}
\fancyhead{}
\lhead{陈翔, 3230103619}
\chead{数据结构与算法第五次作业}
\rhead{Oct.30th, 2024}

\section{实现函数的设计思路}
按照提示,我编写了 \textbf{\texttt{detachrightMin}} 函数。该函数的功能是从给定的 \texttt{t} 节点出发,找出以 \texttt{t} 为根的右子树中的最小节点,返回该节点,并从原子树中断开这个节点。

该函数主要处理以下情况:
1. 当给定节点的右子节点即为右子树的最小节点时,单独设计了断开和连接节点的操作。
2. 通过左遍历找到最小节点。考虑到断开操作可能导致当前节点丢失,我采用了循环和\texttt{current}指针来保留当前节点的信息,实现了 \textbf{findMin} 的功能,并在前后节点之间重新建立连接。

由于最小节点没有左节点,逻辑上仅需考虑上述一种常规情况和一种特殊情况。

\section{测试函数的设计思路}
根据代码逻辑,我保留了已有的测试功能,并单独针对 "给定节点的右子节点无左子节点" 的情况进行了测试。因此,在 \texttt{remove(10)} 之后,我又调用 \texttt{remove(15)} 并打印了树,所有测试结果均正常。
\section{测试的结果}
以下是程序的输出内容:
\begin{verbatim}
Initial Tree:
3
5
7
10
12
15
18
Minimum element: 3
Maximum element: 18
Contains 7? Yes
Contains 20? No
Tree after removing 7:
3
5
10
12
15
18
Tree after removing 10:
3
5
12
15
18
Tree after removing 15:
3
5
12
18
\end{verbatim}

此外,我用 \texttt{valgrind} 进行了内存测试,命令如下:
未检测到内存泄露。

\section{Bug 报告}
在 \texttt{remove} 函数中暂未发现明显问题。

\end{document}
