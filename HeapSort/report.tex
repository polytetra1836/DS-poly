\documentclass[UTF8]{ctexart}
\usepackage{geometry}
\geometry{margin=1.5cm, vmargin={0pt,1cm}}
\setlength{\topmargin}{-1cm}
\setlength{\paperheight}{29.7cm}
\setlength{\textheight}{25.3cm}

% useful packages.
\usepackage{amsfonts}
\usepackage{amsmath}
\usepackage{amssymb}
\usepackage{amsthm}
\usepackage{enumerate}
\usepackage{graphicx}
\usepackage{multicol}
\usepackage{fancyhdr}
\usepackage{layout}
\usepackage{listings}
\usepackage{float, caption}

\lstset{
    basicstyle=\ttfamily, basewidth=0.5em
}

% some common command
\newcommand{\dif}{\mathrm{d}}
\newcommand{\avg}[1]{\left\langle #1 \right\rangle}
\newcommand{\difFrac}[2]{\frac{\dif #1}{\dif #2}}
\newcommand{\pdfFrac}[2]{\frac{\partial #1}{\partial #2}}
\newcommand{\OFL}{\mathrm{OFL}}
\newcommand{\UFL}{\mathrm{UFL}}
\newcommand{\fl}{\mathrm{fl}}
\newcommand{\op}{\odot}
\newcommand{\Eabs}{E_{\mathrm{abs}}}
\newcommand{\Erel}{E_{\mathrm{rel}}}

\begin{document}

\pagestyle{fancy}
\fancyhead{}
\lhead{陈翔, 3230103619}
\chead{数据结构与算法第7次作业}
\rhead{Nov.28th, 2024}

\section{HeapSort的排序实现}
我采用二叉堆的排序算法,其实现的主要思路、功能和函数等有:
\subsection{堆操作}
左式堆的基本操作包括:
\begin{itemize}
    \item \textbf{deleteMin}: 移除并返回堆中的最小元素。对该最小堆而言就是删除根节点,将子节点合并
    \item \textbf{buildHeap}: 从一个无序的向量列表构建堆,通过循环和插入元素实现
    \item \textbf{insert}: 向堆中插入元素,通过合并构建单节点堆和合并实现
\end{itemize}
左式堆的核心操作是\texttt{merge}函数,它通过维持左堆结构来合并两个堆,合并在右侧进行,
降低了合并的操作复杂度。实现步骤有:
\begin{itemize}
    \item \textbf{1}: 合并操作有空节点,则返回平凡的结果
    \item \textbf{2}: 找出根节点更小的堆,如果是h2则和h1交换,用它作为主元,将h2和h1的右子树合并,进入递归
    \item \textbf{3}: 维护h1的左堆结构,如果变成了右堆则交换左右子节点
    \item \textbf{4}:更新h1的npl,返回h1
\end{itemize}
\subsection{Sort排序操作}
读入一个向量,用buildHeap将其构建为堆,然后通过循环和deleteMin操作不断输出最小元到向量中,
该向量就是排序后的元素序列
\section{测试函数的实现}
\subsection{check函数}
从向量中依次读入元素,该向量是顺序排列的,当且仅当每一个元素不超过后一元素,通过循环和反复判断,输出判断结果
\subsection{测试向量}
依次生成5种测试向量,它们的类型都是\textbf{int},长度都是\textbf{size}=1000000:
\begin{itemize}
    \item \textbf{ randomVec}: 随机不重复序列,用STL的\textbf{set}功能保证不重复性,取值从1到size
    \item \textbf{sortVec}: 顺序序列,从0递增到size-1
    \item \textbf{reverseVec}: 逆序序列,从size-1递减至0
    \item \textbf{ duplicateVec}:随机有重复序列,取值从1到size
    \item \textbf{ repeatVec}:全同序列,取值都是size
\end{itemize}
\subsection{测试函数}
对上述5种序列,依次读入,用Sort函数排序,生成排序后的向量,记录排序时间,
用check函数判定排序正确性,输出排序时间和判定结果,
然后用std::sort\_heap排序,记录排序时间并输出。
\subsection{测试环境}
测试结果给出的是\textbf{std=c++20 -Wall -O2}编译指令下的运行时间,同时还进行了非优化环境
的时间测试。经比对,编译优化对std::sort\_heap的排序时间优化更明显,对排序时间较低的优化效果更明显
\section{测试的结果}
\begin{table}[H]
    \centering
    \begin{tabular}{|c|c|c|}
        \hline
        type & HeapSort time (seconds) & std::sort\_heap time (seconds) \\
        \hline
        随机不重复序列 &  0.912465 & 0.0969539 \\
        \hline
        顺序序列 &  0.911185 & 0.100685 \\
        \hline
        逆序序列  & 0.0747315 & 0.11675 \\
        \hline
        随机重复序列 & 1.336 & 0.227278 \\
        \hline
        全同序列 & 0.0678918 & 0.0703563 \\
        \hline
    \end{tabular}
    \caption{排序时间结果比较:关于 HeapSort 和 std::sort\_heap}
    \label{tab:performance}
\end{table}
\section{测试结果分析}
理论上说,二者的时间复杂度都是\textbf{O(NlogN)},因为单步操作的最坏情况是\textbf{O(logN)},而这样的步数
可能有\(\theta\)(N)步,这取决于堆的高度。\par
经过运行时间点测试,发现std::sort\_heap的堆建立时间明显低于HeapSort,这是由二者结构决定的,
尽管该排序算法中,堆建立时间往往小于删除排序时(逆序序列中二者接近相等)
不过我们可以从中验证的结构,结合实现功能,我们知道后者的实现是完全二叉堆,前者是偏重的左堆,
在非逆序的情况下,建堆时间前者高于后者\par
由于std::sort\_heap使用的完全二叉堆结构,在其排序过程中,顺序、逆序和随机不重复序列的运行时间是相近
的。对于输入序列的序关系是相对稳定的。\par
而对于自行设计的HeapSort而言,我们存在以下问题:一方面,维持左堆结构需要交换、判定和更新npl等操作,
这会使得,在通常情况下,效率降低了常数倍C,如同我们看的测试结果一样;与此同时,当数据是明显离散无序的时候,
保持的左堆结构在大概率下会近似于完全堆,其高度与完全堆相差不大,不妨记为k,而维护操作明显降低了其效率,
算法时间复杂度近似为\textbf{CN(logN-k)},相较于\textbf{N(logN)}差了C倍,其中k和c都是大于1的小正常数。\par
\end{document}
