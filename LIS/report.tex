\documentclass{article}
\usepackage{amsmath, amssymb, amsthm, geometry, algorithmicx, algpseudocode}
\usepackage{algorithm}
\usepackage{amsfonts} % 用于数学字体
\usepackage{amssymb} % 用于数学符号
\usepackage{ctex}% 用于中文格式

\begin{document}

\title{最长严格递增子序列动态规划算法}
\author{}
\date{}
\maketitle

\section{问题描述}
给定两个序列 $A = [a_1, a_2, \dots, a_m]$ 和 $B = [b_1, b_2, \dots, b_n]$,找出它们的最长严格递增子序列(LIS)。

\section{动态规划算法}
\subsection{状态定义}
设 $dp[i][j]$ 表示序列 $A[1..i]$ 和 $B[1..j]$ 中的最长严格递增子序列的长度。

\subsection{状态转移方程}
\begin{itemize}
    \item 如果 $A[i] = B[j]$,则最优解至少包含它们中的一个,若同时包含,则删去改2个元素后得到了
    输入为(i-1)(j-1)以及输入元素是该问题子集的最优解,因此有递推关系 ;
    若不是同时包含,不妨设最优解中B不含第j个元素,则该元素在j之前已经有出现,否则与子序列相同矛盾,
    则将B中B[j]最后一次出现的该元素与B[j]替换,则仍是最优解,因此仍满足递推关系
    \[ 
        dp[i][j] = dp[i-1][j-1] + 1 
        \]
    \item 如果 $A[i] \neq B[j]$,最优解不同时包含这2个元素,则删去不含的一个,子序列长度不变,仍是缩小规模
   输入的最优解,因此有递推关系
    \[
    dp[i][j] = \max(dp[i-1][j], dp[i][j-1]) 
    \]
\end{itemize}

\subsection{初始条件}
\[ 
dp[i][0] = 0 \quad (\forall i) 
\]
\[ 
dp[0][j] = 0 \quad (\forall j) 
\]
\begin{algorithm}
    \caption{填表过程}
\begin{algorithmic}[1]
    \State \textbf{输入}:序列 $A$ 和 $B$,它们的长度分别为 $m$ 和 $n$,序列内读入每个元素
    \State \textbf{初始化}:二维数组 $dp$,大小为 $(m+1) \times (n+1)$,全部元素初始化为 0
    \For{$i = 1$ to $m$}
        \For{$j = 1$ to $n$}
            \If{$A[i] = B[j]$}
                \State $dp[i][j] = dp[i-1][j-1] + 1$
            \Else
                \State $dp[i][j] = \max(dp[i-1][j], dp[i][j-1])$
            \EndIf
        \EndFor
    \EndFor
    \State \textbf{输出}:$dp[m][n]$,最长严格递增子序列的长度
    \end{algorithmic}
    \end{algorithm}
最长严格递增子序列的长度为 $dp[m][n]$。

\section{算法复杂度}
该算法的时间复杂度为:
\[
O(m \cdot n)
\]

\section{示例}
令 $A = [a, b, c, b, d, a]$ 和 $B = [b, c, a, b, d]$。

\subsection{初始化}
\[
dp =
\begin{bmatrix}
0 & 0 & 0 & 0 & 0 & 0 \\
0 & 0 & 0 & 0 & 0 & 0 \\
0 & 0 & 0 & 0 & 0 & 0 \\
0 & 0 & 0 & 0 & 0 & 0 \\
0 & 0 & 0 & 0 & 0 & 0 \\
0 & 0 & 0 & 0 & 0 & 0 \\
\end{bmatrix}
\]

\subsection{填表过程}
\subsection*{动态规划表更新过程}

\subsubsection*{初始化状态}
\[
\begin{array}{|c|c|c|c|c|c|c|}
\hline
    & j=0 & j=1 & j=2 & j=3 & j=4 & j=5 \\ \hline
i=0 & 0   & 0   & 0   & 0   & 0   & 0   \\ \hline
i=1 & 0   & 0   & 0   & 0   & 0   & 0   \\ \hline
i=2 & 0   & 0   & 0   & 0   & 0   & 0   \\ \hline
i=3 & 0   & 0   & 0   & 0   & 0   & 0   \\ \hline
i=4 & 0   & 0   & 0   & 0   & 0   & 0   \\ \hline
i=5 & 0   & 0   & 0   & 0   & 0   & 0   \\ \hline
i=6 & 0   & 0   & 0   & 0   & 0   & 0   \\ \hline
\end{array}
\]


\subsubsection*{更新第1列,当i=2,j=1发生增长,对应与第一个b匹配)}
\[
\begin{array}{|c|c|c|c|c|c|c|}
\hline
    & j=0 & j=1 & j=2 & j=3 & j=4 & j=5 \\ \hline
i=0 & 0   & 0   & 0   & 0   & 0   & 0   \\ \hline
i=1 & 0   & 0   & 0   & 0   & 0   & 0   \\ \hline
i=2 & 0   & 1   & 0   & 0   & 0   & 0   \\ \hline
i=3 & 0   & 1   & 0   & 0   & 0   & 0   \\ \hline
i=4 & 0   & 1   & 0   & 0   & 0   & 0   \\ \hline
i=5 & 0   & 1   & 0   & 0   & 0   & 0   \\ \hline
i=6 & 0   & 1   & 0   & 0   & 0   & 0   \\ \hline
\end{array}
\]

\subsubsection*{更新第2列,i=3,j=2发生增长,对应于c发生匹配}
\[
\begin{array}{|c|c|c|c|c|c|c|}
\hline
    & j=0 & j=1 & j=2 & j=3 & j=4 & j=5 \\ \hline
i=0 & 0   & 0   & 0   & 0   & 0   & 0   \\ \hline
i=1 & 0   & 0   & 0   & 0   & 0   & 0   \\ \hline
i=2 & 0   & 1   & 1   & 0   & 0   & 0   \\ \hline
i=3 & 0   & 1   & 2   & 0   & 0   & 0   \\ \hline
i=4 & 0   & 1   & 2   & 0   & 0   & 0   \\ \hline
i=5 & 0   & 1   & 2   & 0   & 0   & 0   \\ \hline
i=6 & 0   & 1   & 2   & 0   & 0   & 0   \\ \hline
\end{array}
\]

\subsubsection*{更新第3列,i=1,j=3;i=1,j=6发生增长,对应于a的匹配}
\[
\begin{array}{|c|c|c|c|c|c|c|}
\hline
    & j=0 & j=1 & j=2 & j=3 & j=4 & j=5 \\ \hline
i=0 & 0   & 0   & 0   & 0   & 0   & 0   \\ \hline
i=1 & 0   & 0   & 0   & 1   & 0   & 0   \\ \hline
i=2 & 0   & 1   & 1   & 1   & 0   & 0   \\ \hline
i=3 & 0   & 1   & 2   & 2   & 0   & 0   \\ \hline
i=4 & 0   & 1   & 2   & 2   & 0   & 0   \\ \hline
i=5 & 0   & 1   & 2   & 2   & 0  & 0   \\ \hline
i=6 & 0   & 1   & 2   & 3   & 0   & 0   \\ \hline
\end{array}
\]
\subsubsection*{更新第3列,i=2,j=4;i=1,j=6发生增长,分别对应A的第一个、第二个b和B中的元素b的匹配}
\[
\begin{array}{|c|c|c|c|c|c|c|}
\hline
    & j=0 & j=1 & j=2 & j=3 & j=4 & j=5 \\ \hline
i=0 & 0   & 0   & 0   & 0   & 0   & 0   \\ \hline
i=1 & 0   & 0   & 0   & 1   & 1   & 0   \\ \hline
i=2 & 0   & 1   & 1   & 1   & 2   & 0   \\ \hline
i=3 & 0   & 1   & 2   & 2   & 3   & 0   \\ \hline
i=4 & 0   & 1   & 2   & 2   & 3  & 0   \\ \hline
i=5 & 0   & 1   & 2   & 2   & 3  & 0   \\ \hline
i=6 & 0   & 1   & 2   & 3   & 3   & 0   \\ \hline
\end{array}
\]
\subsubsection*{更新第5列,i=4,j=5,对应于D的匹配}
\[
\begin{array}{|c|c|c|c|c|c|c|}
\hline
    & j=0 & j=1 & j=2 & j=3 & j=4 & j=5 \\ \hline
i=0 & 0   & 0   & 0   & 0   & 0   & 0   \\ \hline
i=1 & 0   & 0   & 0   & 1   & 1   & 1   \\ \hline
i=2 & 0   & 1   & 1   & 1   & 2   & 2   \\ \hline
i=3 & 0   & 1   & 2   & 2   & 3   & 3   \\ \hline
i=4 & 0   & 1   & 2   & 2   & 3   & 4   \\ \hline
i=5 & 0   & 1   & 2   & 2   & 3   & 4   \\ \hline
i=6 & 0   & 1   & 2   & 3   & 3   & 4  \\ \hline
\end{array}
\]

\subsection*{最终结果}


\subsection{最终结果}
最长严格递增子序列长度为:
\[
dp[6][5] =4
\]

\end{document}
